\documentclass[xcolor=dvipsnames,pdftex]{beamer}

\mode<presentation>
{
	%\usetheme{Warsaw}
	\usetheme{default}
	\usecolortheme{albatross}
	\usefonttheme{structurebold}

	%\setbeamercovered{transparent}
}


%\usepackage[english]{babel}
\usepackage[ngerman]{babel}

%\usepackage[latin1]{inputenc}
\usepackage[utf8x]{inputenc}

\usepackage{times}
%\usepackage{pxfonts}
%\usepackage{lmodern}
\usepackage[T1]{fontenc}
% Or whatever. Note that the encoding and the font should match. If T1
% does not look nice, try deleting the line with the fontenc.

%\usepackage{xspace}
%\usepackage{mathpartir}
%\usepackage{stmaryrd}
%\usepackage{enumerate}
%\usepackage{cancel}
%\usepackage{amsmath}
%\usepackage{tikz}

\usepackage{listings}

%\lstset{
%language=[ANSI]C,
%basicstyle=\ttfamily,
%keywordstyle=\color{green}\textbf,
%stringstyle=\color{red},
%commentstyle=\color{cyan}\textit,
%tabsize=2,
%tab=\rightarrowfill
%}

\lstset{
language=bash,
tabsize=2,
tab=\rightarrowfill
}

\hypersetup{
	pdfpagemode=FullScreen,
	colorlinks=true,
	urlcolor=SkyBlue,
	linkcolor=orange
}

\pdfcompresslevel=9



%\setbeamertemplate{navigation symbols}%[only frame symbol]
%{
%}


\setbeamerfont{page number in head/foot}{size=\footnotesize}

\setbeamertemplate{footline}[frame number]
{
	%\insertframenumber/\inserttotalframenumber
}

\title % [Short Paper Title] % (optional, use only with long paper titles)
{BadUSB}

%\subtitle
%{RTHAL versus ADEOS} % (optional)

%\author[Stephan Wendt] % (optional, use only with lots of authors)
%{Stephan Wendt\\
%\href{mailto:stephanwendt@freenet.de}{stephanwendt@freenet.de}}
% - Use the \inst{?} command only if the authors have different
%   affiliation.
\author{Michael Koch\\
\href{mailto:pib.michael.koch@htw-saarland.de}{pib.michael.koch@htw-saarland.de}\\
\and{Urs Oberdorf\\
\href{mailto:urs.oberdorf@autistici.org}{urs.oberdorf@autistici.org}}\\
\and{Stephan Wendt\\
\href{mailto:stephanwendt@freenet.de}{stephanwendt@freenet.de}}}


\institute %[Universities of Somewhere and Elsewhere] % (optional, but mostly needed)
{
	Fakultät für Ingenieurwissenschaften\\
	Hochschule für Technik und Wirtschaft des Saarlandes
}
% - Use the \inst command only if there are several affiliations.
% - Keep it simple, no one is interested in your street address.

\date  %[qsx05]  (optional)
{06.07.2015}

%\subject{CDuce: An XML-Centric General-Purpose Language}
% This is only inserted into the PDF information catalog. Can be left
% out. 



%\logo{\includegraphics[width=\linewidth]{../Graphics/htwsaar_Logo_inwi_head_VF_4C_crop}}



% Delete this, if you do not want the table of contents to pop up at
% the beginning of each subsection:
%\AtBeginSubsection[]
%{
%  \begin{frame}<beamer>
%    \frametitle{Outline}
%    \tableofcontents[currentsection,currentsubsection]
%  \end{frame}
%}

%\AtBeginSection[] % Do nothing for \section*
%{
%	\begin{frame}<beamer>
%		\frametitle{Gliederung}
%		\tableofcontents[currentsection]
%	\end{frame}
%}


% If you wish to uncover everything in a step-wise fashion, uncomment
% the following command: 

%\beamerdefaultoverlayspecification{<+->}

\begin{document}

\begin{frame}
	\titlepage
\end{frame}



% Since this a solution template for a generic talk, very little can
% be said about how it should be structured. However, the talk length
% of between 15min and 45min and the theme suggest that you stick to
% the following rules:  

% - Exactly two or three sections (other than the summary).
% - At *most* three subsections per section.
% - Talk about 30s to 2min per frame. So there should be between about
%   15 and 30 frames, all told.


%\subsection[Short First Subsection Name]{First Subsection Name}

%-------------------------------------------------------------------- Frame --
\begin{frame}{BadUSB}{Einleitung}
	\begin{itemize}
		\item \glqq Integrierte Entwicklungsumgebung\grqq (\textbf{I}ntegrated \textbf{D}evelopment
			\textbf{E}nvironment)
			\pause
		\item Vollständig in Java implementiert
			\pause
		\item Weitgehend plattformunabhängig
			\pause
		\item Modular fast beliebig erweiterbar
	\end{itemize}
\end{frame}

%-------------------------------------------------------------------- Frame --
\begin{frame}{Netbeans}{Einleitung}
	Unterstützte Sprachen:
	\bigskip
	\begin{itemize}
		\item HTML, JavaScript, CSS
			\pause
		\item Java, C, C++, Fortran, PHP, etc.
			\pause
		\item beliebige weitere über Plugins
	\end{itemize}
\end{frame}

%-------------------------------------------------------------------- Frame --
\begin{frame}{Netbeans}{Einleitung}
	Funktionsumfang u.a.:
	\begin{itemize}
		\item Quellcode editieren:\\
			Syntax highlighting, intelligente Code-Vervollständigung,
			Refactoring, Code-Skeletons, Navigieren zwischen
			Sprachelementen etc.
			\pause
		\item Projekt kompilieren\\
			mit allen Abhängigkeiten, Maven etc.
			\pause
		\item Projekt laufen lassen
			\pause
		\item Projekt debuggen
			\pause
		\item Projekt testen
	\end{itemize}
\end{frame}

%-------------------------------------------------------------------- Frame --
\begin{frame}{Netbeans}{Installation}
	\begin{itemize}
		\item Für HTML und JavaScript-Unterstützung mindestens Version 7.3
			\pause
		\item In Debian- und Ubuntu-Repositories nur Version 7.0.1
			\pause
		\item Benötigt mindestens JDK 7
			\pause
		\item Download unter\\
			\href{https://netbeans.org/downloads/}{https://netbeans.org/downloads/}\\
			bzw. inklusive JDK unter\\
			\href{http://www.oracle.com/technetwork/java/javase/downloads/jdk-netbeans-jsp-142931.html}{http://www.oracle.com/technetwork/java/javase/downloads/jdk-netbeans-jsp-142931.html}
			\pause
		\item Installer unter Linux und Windows ungefähr gleich intuitiv benutzbar
	\end{itemize}
\end{frame}

%-------------------------------------------------------------------- Frame --
\begin{frame}{Netbeans}{Web-Projekt entwickeln}
	Netbeans-Extension für den Webbrowser Chrome installieren:\\
	\href{https://chrome.google.com/webstore/search/netbeans}{https://chrome.google.com/webstore/search/netbeans} -> "NetBeans Connector"
\end{frame}

%-------------------------------------------------------------------- Frame --
\begin{frame}{Netbeans}{Web-Projekt entwickeln}
	\begin{itemize}
		\item File -> New Project -> Categories: HTML5 -> Projects: HTML5 Application -> Next\\
		\item Projektname und Projektpfade wählen -> Next\\
		\item Templates wählen -> Next\\
		\item JavaScript-Bibliotheken einbinden\\
	\end{itemize}
\end{frame}

\begin{frame}{Netbeans}{Web-Projekt entwickeln}
	%%TODO: Screenshot
\end{frame}

\begin{frame}{Netbeans}{Web-Projekt entwickeln}
	\begin{itemize}
		\item Code editieren im Source Editor-Fenster
			(wenn Code-Vervollständigung nicht automatisch erscheint,
			mit Tastenkombination Ctrl-Spacebar aufrufen)
			\pause
		\item Im Code navigieren im Navigator-Fenster
			\pause
		\item Dateien (CSS, JavaScript, Bilder, etc.) dem Projekt hinzufügen im Projects-Fenster
			\pause
		\item Run -> Run Project: Projekt ausführen und im Browser anzeigen
			lassen
	\end{itemize}
\end{frame}



\begin{frame}{Netbeans}{Web-Projekt debuggen}
	\begin{itemize}
		\item In JavaScript-Dateien können am linken Rand im Code-Editor (bei den
			Zeilennummern) Breakpoints gesetzt werden\\
		\item Ausführung des Projekts stoppt am Breakpoint\\
		\item Variablen-Inhalte können dann über Tooltips der Variablen im Code oder
			im Variablen-Fenster angesehen werden\\
		\item Code-Abschnitte können markiert werden und über Debug -> Evaluate
			Expression ausgeführt werden\\
		\item (Window -> Debugging -> Breakpoints)\\
			(Window -> Debugging -> Variables)\\
			...
	\end{itemize}
\end{frame}


\begin{frame}{Netbeans}{Unit-Tests}
	\begin{itemize}
		\item JsTestDriver-1.3.5.jar von\\
			\href{https://code.google.com/p/js-test-driver/downloads/list}{https://code.google.com/p/js-test-driver/downloads/list}\\
			downloaden\\
		\item Window -> Services -> JS Test Driver -> Configure\\
		\item JsTestDriver JAR: heruntergeladene JAR-Datei\\
			Browsers to use for testing:\\
			Chrome with Netbeans Connector
	\end{itemize}
\end{frame}


\begin{frame}{Netbeans}{Unit-Tests}
	\begin{itemize}
		\item Rechtsklick auf das Projekt -> New -> Other...\\
		\item Categories: Unit Tests -> File Types: jsTestDriver Configuration File -> Next\\
		\item Finish\\
		\item In der nun geöffneten jsTestDriver.conf Datei können unter \texttt{load:}
			die Unit-Test-Dateien und die zu testenden Dateien eingetragen werden\\
		\item Rechtsklick auf das Projekt -> Test, um die Tests zu starten
	\end{itemize}
\end{frame}

\begin{frame}{Netbeans}{Refactoring}
	Über Menü \textit{Refactor}
\end{frame}

\begin{frame}{Netbeans}{Versionsverwaltung}
	Team -> Git -> Clone...\\
	Repository URL:, User: und Password: eingeben
\end{frame}

\begin{frame}{Ende}
	\begin{center}
		\Large{Fragen?}
	\end{center}
\end{frame}

\end{document}
