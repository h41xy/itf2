\documentclass[xcolor=dvipsnames,pdftex]{beamer}

\mode<presentation>
{
	%\usetheme{Warsaw}
	\usetheme{default}
	\usecolortheme{albatross}
	\usefonttheme{structurebold}

	%\setbeamercovered{transparent}
}


%\usepackage[english]{babel}
\usepackage[ngerman]{babel}

%\usepackage[latin1]{inputenc}
\usepackage[utf8x]{inputenc}

\usepackage{times}
%\usepackage{pxfonts}
%\usepackage{lmodern}
\usepackage[T1]{fontenc}
% Or whatever. Note that the encoding and the font should match. If T1
% does not look nice, try deleting the line with the fontenc.

%\usepackage{xspace}
%\usepackage{mathpartir}
%\usepackage{stmaryrd}
%\usepackage{enumerate}
%\usepackage{cancel}
%\usepackage{amsmath}
%\usepackage{tikz}

\usepackage{listings}

%\lstset{
%language=[ANSI]C,
%basicstyle=\ttfamily,
%keywordstyle=\color{green}\textbf,
%stringstyle=\color{red},
%commentstyle=\color{cyan}\textit,
%tabsize=2,
%tab=\rightarrowfill
%}

\lstset{
language=bash,
tabsize=2,
tab=\rightarrowfill
}

\hypersetup{
	pdfpagemode=FullScreen,
	colorlinks=true,
	urlcolor=SkyBlue,
	linkcolor=orange
}

\pdfcompresslevel=9



%\setbeamertemplate{navigation symbols}%[only frame symbol]
%{
%}


\setbeamerfont{page number in head/foot}{size=\footnotesize}

\setbeamertemplate{footline}[frame number]
{
	%\insertframenumber/\inserttotalframenumber
}

\title % [Short Paper Title] % (optional, use only with long paper titles)
{BadUSB}

%\subtitle
%{RTHAL versus ADEOS} % (optional)

%\author[Stephan Wendt] % (optional, use only with lots of authors)
%{Stephan Wendt\\
%\href{mailto:stephanwendt@freenet.de}{stephanwendt@freenet.de}}
% - Use the \inst{?} command only if the authors have different
%   affiliation.
\author{Michael Koch\\
\href{mailto:pib.michael.koch@htw-saarland.de}{pib.michael.koch@htw-saarland.de}\\
\and{Urs Oberdorf\\
\href{mailto:urs.oberdorf@autistici.org}{urs.oberdorf@autistici.org}}\\
\and{Stephan Wendt\\
\href{mailto:stephanwendt@freenet.de}{stephanwendt@freenet.de}}}


\institute %[Universities of Somewhere and Elsewhere] % (optional, but mostly needed)
{
	Fakultät für Ingenieurwissenschaften\\
	Hochschule für Technik und Wirtschaft des Saarlandes
}
% - Use the \inst command only if there are several affiliations.
% - Keep it simple, no one is interested in your street address.

\date  %[qsx05]  (optional)
{06.07.2015}

%\subject{CDuce: An XML-Centric General-Purpose Language}
% This is only inserted into the PDF information catalog. Can be left
% out. 



%\logo{\includegraphics[width=\linewidth]{../Graphics/htwsaar_Logo_inwi_head_VF_4C_crop}}



% Delete this, if you do not want the table of contents to pop up at
% the beginning of each subsection:
%\AtBeginSubsection[]
%{
%  \begin{frame}<beamer>
%    \frametitle{Outline}
%    \tableofcontents[currentsection,currentsubsection]
%  \end{frame}
%}

%\AtBeginSection[] % Do nothing for \section*
%{
%	\begin{frame}<beamer>
%		\frametitle{Gliederung}
%		\tableofcontents[currentsection]
%	\end{frame}
%}


% If you wish to uncover everything in a step-wise fashion, uncomment
% the following command: 

%\beamerdefaultoverlayspecification{<+->}

\begin{document}

\begin{frame}
	\titlepage
\end{frame}



% Since this a solution template for a generic talk, very little can
% be said about how it should be structured. However, the talk length
% of between 15min and 45min and the theme suggest that you stick to
% the following rules:  

% - Exactly two or three sections (other than the summary).
% - At *most* three subsections per section.
% - Talk about 30s to 2min per frame. So there should be between about
%   15 and 30 frames, all told.


%\subsection[Short First Subsection Name]{First Subsection Name}

%-------------------------------------------------------------------- Frame --
\begin{frame}{BadUSB}{Einleitung}
	\begin{itemize}
		\item Entwurf in den 90ern
        \item bietet hohe Anschluss-Flexibilität bei Peripheriegeräten
        \item Bus-System
        \item Host-Controller kontrolliert Kommunikation
	\end{itemize}
\end{frame}

%-------------------------------------------------------------------- Frame --
\begin{frame}{BadUSB}{Anschluss eines Geräts}
	\begin{itemize}
        \item Reset
        \item Device Descriptor
        \item Interface Descriptor
	\end{itemize}
\end{frame}
%%-------------------------------------------------------------------- Frame --
\begin{frame}{BadUSB}{Live Demonstration}
	\begin{itemize}
        \item Live Demonstration
	\end{itemize}
\end{frame}
%-------------------------------------------------------------------- Frame --
\begin{frame}{BadUSB}{Vorgehensweise}
    Testumgebung
	\begin{itemize}
        \item Notebook mit Windows 7
        \item HP Proliant DL380 G5, 2x Xeon Quadcore, 16GB RAM
	\end{itemize}
\end{frame}
%-------------------------------------------------------------------- Frame --
\begin{frame}{BadUSB}{Analyse}
    Drei Images (base, hello, ftp)
	\begin{itemize}
        \item Timeline
        \item MD5 Hashes der Dateien 
        \item Differenzen
        \item Windows Logfiles
        \item setupapi.dev.log
	\end{itemize}
\end{frame}
%-------------------------------------------------------------------- Frame --
\begin{frame}{BadUSB}{Fazit}
    \begin{centering}
        Fazit
    \end{centering}
\end{frame}

%-------------------------------------------------------------------- Frame --
\begin{frame}{BadUSB}{Ausblick}
    \begin{itemize}
        \item Abwehrmöglichkeiten
    \end{itemize}
\end{frame}

%-------------------------------------------------------------------- Frame --
\begin{frame}{BadUSB}{Ende}
    Vielen Dank für Ihre Aufmerksamkeit
\end{frame}





\end{document}
