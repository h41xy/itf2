\section{hashdeep mit Linux}
Nachdem die Festplattenimages mit \textit{dd} erstellt und ge\textit{mount}et wurden, muss nun verglichen werden was sich geändert hat.
Dazu wird von jeder Datei des Festplattenimages die MD5 Summe erstellt und in eine Datei geschrieben.
Dieser Vorgang wird für die Festplattenimages nach der Kompromitierung wiederholt.
Daraufhin vergleicht man die MD5 / Dateinamen Paare der beiden Dateien und erhält eine Liste derer die verändert wurden.

Vorrausgesetzt die Images sind nach /mnt/base, /mnt/hello und /mnt/ftp gemounted lauten die Konsoleneingaben wie folgt:
\begin{lstlisting}
cd /mnt/base
hashdeep -r -l -e -o f . > /var/tmp/base_hashdeep.txt
\end{lstlisting}
\underline{Hinweis:} Die relative Pfadangabe funktioniert nur wenn man sich im Stammverzeichniss der zu prüfenden Dateistruktur befindet.\\[3mm]
Die Schalter bedeuten rekursiv(-r), relative Pfadangabe(-l), (Datei-)fortschritt(-e) und die Option nur normale Dateien zu bearbeiten(-o f).
In Unserem Fall blieb der Prozess bei ca. 30\% hängen ohne die Option nur normale Dateien zu bearbeiten.
Wir haben nicht weiter daran geforscht, vermuten aber dass der Prozess versucht die MD5 Summe einer \textit{pipe} zu berechnen und durch das Fehlen einer EOF nicht weitermacht.

Da \textit{hashdeep} mehrere Threads startet ist die Reihenfolge der \textit{MD5 / Dateinamen} Paare nicht immer die selbe, deshalb haben wir eine anschliessende Sortierung nach Dateinamen vorgenommen.
\begin{lstlisting}
sort -t ',' -k 4 /var/tmp/base_hashdeep.txt > /var/tmp/basesort_hashdeep.txt
\end{lstlisting}

Die obigen Schritte müssen für alle Images durchgeführt werden.

Ist dies erfolgt, so koennen die Ausgabedateien nun auf Unterschiede überprüft werden.
\begin{lstlisting}
diff basesort_hashdeep.txt hellosort_hashdeep.txt > diff_base_hello.txt
\end{lstlisting}
Die Ausgabe kann noch  durch 
\begin{lstlisting}
cut -d ',' -f 1,4 
\end{lstlisting}
geleitet werden um ein wenig mehr Übersicht zu erhalten.
Für ein weiteres Maß an Übersicht kann es Anschliessend noch durch
\begin{lstlisting}
sort -t ',' -f 2 
\end{lstlisting}
geleitet werden um Dateigrößenunterschiede schneller zu erkennen.
