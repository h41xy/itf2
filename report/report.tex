\documentclass[
a4paper, %default
twoside%,
%twocolumn%,
%12pt,
%landscape
%portrait %default
%draft
]{article}
%\special{landscape}
%\pagenumbering{style}
%%style: arabic roman alph Roman Alph

\usepackage{fullpage} %% 1 Zoll Ränder
%\documentclass{proc}
%\documentclass[xcolor=dvipsnames]{beamer}


\usepackage[utf8x]{inputenc}
%\usepackage[latin1]{inputenc}

%\usepackage[ngerman]{babel}
\usepackage[english]{babel}

%% Fonts
\usepackage{times}    %% Times
%\usepackage{pxfonts} %% Palatino
%\usepackage{lmodern} %% Computer Modern (mit T1-Encoding)
\usepackage[T1]{fontenc} %%Option T1 für fontenc-Makro bei PDFLTEX nicht möglich
%\usepackage{cmap}
%Und wenn du noch pslatex einbindest, sehen die Dokumente noch etwas
%hübscher aus

%\usepackage{pstricks,pst-node,pst-text,pst-3d}
\usepackage{amsmath}
\usepackage[pdftex, colorlinks, linkcolor=blue, urlcolor=blue]{hyperref}
%\hypersetup{colorlinks=true, linkcolor=cyan, urlcolor=cyan}
%\hypersetup{
%	pdftex,
%	pdfpagemode=FullScreen,
%	colorlinks=true,
%	urlcolor=SkyBlue,
%	linkcolor=orange
%}
\pdfcompresslevel=9

%\usepackage{longtable}  %% Support for tables longer than a page.
%% Generally easier to use and more flexible than supertabular.

%\usepackage{expdlist}
%\usepackage{enumerate} %%flexible (alphanumerische) Nummerierungen

\usepackage{listings} %% pretty printing of source code

\lstset{
language=bash,
basicstyle=\ttfamily,
keywordstyle=\color{green}\textbf,
stringstyle=\color{red},
commentstyle=\color{cyan}\textit,
tabsize=2,
tab=\rightarrowfill
}

\usepackage{graphicx}


\title{\includegraphics[width=\linewidth]{../Graphics/htwsaar_Logo_inwi_head_VF_4C_crop}\\ BadUSB}
\author{Michael Koch
\thanks{pib.michael.koch@htw-saarland.de}
\and{Urs Oberdorf}
\thanks{urs.oberdorf@autistici.org}
\and{Stephan Wendt}
\thanks{stephanwendt@freenet.de}}
%\email{bar@mailinator.com}
%\institution{HTW des Saarlandes}
\date{01.07.2015}

\begin{document}
%\begin{titlepage}
%	\begin{center}
%		{\LARGE \textbf{Mein toller Titel}}
%	\end{center}
%\end{titlepage}

\maketitle
%\includegraphics[width=\linewidth]{../Graphics/htwsaar_Logo_inwi_head_VF_4C_crop}

%\underline{text} %%Unterstrichener Text


%\tableofcontents

%\textwidth, \columnwidth, \linewidth

\section{Introduction}
\subsection{Angriffsmöglichkeiten}
\begin{itemize}
\item Tastatur-Eingaben (bzw. allgemein HID-Devices)
\item "Boot-Sektor-Virus"
\item gefälschter Netzwerkadapter mit DHCP-Server
\item FTP-Server
\item alles verfälschen, was auf dem Stick gespeichert werden soll - Abwehr: Stick komplett mit Truecrypt/LUKS verschlüsseln
\item versteckte Partition
\item Android-Phone mit noch erweiterten Möglichkeiten (z.B. der komplette
	Netzwerktraffic läuft über das Android-Phone
\end{itemize}


\section{USB - Eine kurze Einführung}
In der Mitte der neunziger Jahre entwickelte ein Firmenkonsortium den \textit{Universal Serial Bus}.
Er sollte einen Standart für den Anschluss von Peripheriegeräten bieten und wurde ein Erfolg.
Heute finden sich an nahezu jedem Peripheriegerät eine USB Schnittstelle.

Wie im Namen enthalten, handelt es sich um ein Bus System welches einen Mechanismus benötigt, die Kommunikation zu steuern.
Im Fall des USB Standarts ist das der auf der Hauptplatine verbaute Host-Controller der angeschlossenen Geräten u.a. das Senden von Daten erlaubt.
Das bedeutet, dass ein per USB angeschlossenes Gerät nur dann Daten senden kann wenn es von dem Host-Controller abgefragt wird.

Sobald an einen USB-Port ein Gerät angeschlossen wird, sendet der Host-Controller ein USB-Reset Signal an das Betreffende Gerät.
Dieses wird dadurch aufgefordert sich neu zu konfigurieren und seinen \textit{Device Descriptor} zu senden.
Dieser Deskriptor enthält Informationen über die Klasse des Geräts, also als welches Gerät es sich selbst anmelden möchte sowie welcher Treiber dafür geladen werden soll.
Für das senden dieses Deskriptors ist die Firmware auf dem Chip verantwortlich.
Wenn diese manipuliert wurde kann sich das Gerät, z.B. ein USB-Stick, auch als Tastatur identifizieren und beginnen Eingabesignale zu senden wie eine normale Tastatur.
Die Spezifikation ist ausserdem so ausgelegt, dass sich Geräte auch mit einem \textit{Interface Deskriptor} anmelden können und somit zwei Geräte gleichzeitig sein können.
Gewünscht ist dieser Fall bei z.B. Webcams welche sich als Audio aber auch Video Gerät anmelden.
Dies ist nur ein Beispiel was mit einer Firmwaremanipulation möglich ist.
Alles was per USB angeschlossen werden kann, kann von einem anderen USB-Gerät über seine imitiert werden.


\subsection{Abwehrmöglichkeit}
\begin{itemize}
\item nur USB-Massenspeicher zulassen: darf dann aber nicht beim Reboot drin stecken, bzw. Booten von USB-Sticks unterbinden und BIOS-Passwort setzen, damit der USB-Stick nicht mit Tastatureingaben das BIOS umkonfigurieren kann.
Problem ist aber, irgendeine Tastatur muss zugelassen werden, zum Zeitpunkt
an dem das BIOS den Rechner an das OS übergibt, welche sollte das sein?
Vielleicht nach dem BIOS keine Tastatureingaben zulassen, bis eine Passworteingabe verlangt wird (entweder vom Bootloader oder vom Betriebssystem) und die Tastatur, die dann das richtige Passwort eingibt, ist die erlaubte Tastatur)
Wie würde man es mit der Maus machen?
Tastatur nur an bestimmtem Port zulassen?
\item Sollte man Bestätigungsdialoge beim Anschließen von neuer
	USB-Hardware einblenden, so dürfen diese nur von Menschen bedienbar
	sein (CAPTCHA).
\item Stick direkt beim Anstecken mit sauberer Firmware überschreiben? - Klappt
  wahrscheinlich nicht, weil das Betriebssystem das auf der USB-Stick-CPU
läuft, wahrscheinlich sich dazwischen schalten kann
\item The firmware of a USB device can typically only be read back with the help of that firmware (if at all): A malicious firmware can spoof a legitimate one. D.h. es würde nichts bringen, die Firmware auszulesen, um sie zu überprüfen.
Ist es irgendwie feststellbar, dass es doch mal ein USB-Stick war, bevor die manipulierte Firmware aktiv wird?
\item Abfragen, ob die Chip-Familie einen Hardware-Firmwareschreibschutz hat.
\item FreeBSD adds an option to switch off USB enumeration.
\item Vielleicht alle erlaubten USB-Geräte so umflashen, dass sie sich eindeutig identifizieren?
\item Offene Hardware und Firmware, und USB-Geräte, die man zuverlässig mit der eigenen signierten Firmware flashen kann
\item Wie kann man verhindern, dass die eigenen USB-Geräte (Sticks, etc.),
	die man an fremde Rechner anschließt, infiziert werden?
\item Nur Flashspeicher verwenden, der komplett vom Betriebssystem
	gesteuert wird.
\end{itemize}

\section{hashdeep mit Linux}
Nachdem die Festplattenimages mit \textit{dd} erstellt und ge\textit{mount}et wurden, muss nun verglichen werden was sich geändert hat.
Dazu wird von jeder Datei des Festplattenimages die MD5 Summe erstellt und in eine Datei geschrieben.
Dieser Vorgang wird für die Festplattenimages nach der Kompromitierung wiederholt.
Daraufhin vergleicht man die MD5 / Dateinamen Paare der beiden Dateien und erhält eine Liste derer die verändert wurden.

Vorrausgesetzt die Images sind nach /mnt/base, /mnt/hello und /mnt/ftp gemounted lauten die Konsoleneingaben wie folgt:
\begin{lstlisting}
cd /mnt/base
hashdeep -r -l -e -o f . > /var/tmp/base_hashdeep.txt
\end{lstlisting}
\underline{Hinweis:} Die relative Pfadangabe funktioniert nur wenn man sich im Stammverzeichniss der zu prüfenden Dateistruktur befindet.\\[3mm]
Die Schalter bedeuten rekursiv(-r), relative Pfadangabe(-l), (Datei-)fortschritt(-e) und die Option nur normale Dateien zu bearbeiten(-o f).
In Unserem Fall blieb der Prozess bei ca. 30\% hängen ohne die Option nur normale Dateien zu bearbeiten.
Wir haben nicht weiter daran geforscht, vermuten aber dass der Prozess versucht die MD5 Summe einer \textit{pipe} zu berechnen und durch das Fehlen einer EOF nicht weitermacht.

Da \textit{hashdeep} mehrere Threads startet ist die Reihenfolge der \textit{MD5 / Dateinamen} Paare nicht immer die selbe, deshalb haben wir eine anschliessende Sortierung nach Dateinamen vorgenommen.
\begin{lstlisting}
sort -t ',' -k 4 /var/tmp/base_hashdeep.txt > /var/tmp/basesort_hashdeep.txt
\end{lstlisting}

Die obigen Schritte müssen für alle Images durchgeführt werden.

Ist dies erfolgt, so koennen die Ausgabedateien nun auf Unterschiede überprüft werden.
\begin{lstlisting}
diff basesort_hashdeep.txt hellosort_hashdeep.txt > diff_base_hello.txt
\end{lstlisting}
Die Ausgabe kann noch  durch 
\begin{lstlisting}
cut -d ',' -f 1,4 
\end{lstlisting}
geleitet werden um ein wenig mehr Übersicht zu erhalten.
Für ein weiteres Maß an Übersicht kann es Anschliessend noch durch
\begin{lstlisting}
sort -t ',' -f 2 
\end{lstlisting}
geleitet werden um Dateigrößenunterschiede schneller zu erkennen.


\subsection{Projektrichtung}
\begin{itemize}
\item neuartige Attacke im USB-Stick
\item Entwickeln und Testen von Schutzmaßnahmen
\item Versuchen, ein fest in den Laptop eingebautes Gerät zu flashen
\item Versuchen, mit forensischen Methoden einen Angriff per USB-Gerät
  festzustellen
\end{itemize}
\url{http://www.crypto-fuer-alle.de/wishlist/hardware-usb-filter/}
\url{http://theinvisiblethings.blogspot.de/2011/06/usb-security-challenges.html}
\url{https://github.com/adamcaudill/Psychson}
\end{document}
