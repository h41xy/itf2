\documentclass[
a4paper, %default
twoside%,
%twocolumn%,
%12pt,
%landscape
%portrait %default
%draft
]{article}
%\special{landscape}
%\pagenumbering{style}
%%style: arabic roman alph Roman Alph

\usepackage{fullpage} %% 1 Zoll Ränder
%\documentclass{proc}
%\documentclass[xcolor=dvipsnames]{beamer}


\usepackage[utf8x]{inputenc}
%\usepackage[latin1]{inputenc}

%\usepackage[ngerman]{babel}
\usepackage[english]{babel}

%% Fonts
\usepackage{times}    %% Times
%\usepackage{pxfonts} %% Palatino
%\usepackage{lmodern} %% Computer Modern (mit T1-Encoding)
\usepackage[T1]{fontenc} %%Option T1 für fontenc-Makro bei PDFLTEX nicht möglich
%\usepackage{cmap}
%Und wenn du noch pslatex einbindest, sehen die Dokumente noch etwas
%hübscher aus

%\usepackage{pstricks,pst-node,pst-text,pst-3d}
\usepackage{amsmath}
\usepackage[pdftex, colorlinks, linkcolor=blue, urlcolor=blue]{hyperref}
%\hypersetup{colorlinks=true, linkcolor=cyan, urlcolor=cyan}
%\hypersetup{
%	pdftex,
%	pdfpagemode=FullScreen,
%	colorlinks=true,
%	urlcolor=SkyBlue,
%	linkcolor=orange
%}

%\usepackage{longtable}  %% Support for tables longer than a page.
%% Generally easier to use and more flexible than supertabular.

%\usepackage{expdlist}
%\usepackage{enumerate} %%flexible (alphanumerische) Nummerierungen

%\usepackage{listings} %% pretty printing of source code
%
%\lstset{
%language=[ANSI]C,
%basicstyle=\ttfamily,
%keywordstyle=\color{green}\textbf,
%stringstyle=\color{red},
%commentstyle=\color{cyan}\textit,
%tabsize=2,
%tab=\rightarrowfill
%}



\title{Badusb}
\author{Michael Koch
\thanks{pib.michael.koch@htw-saarland.de}
\and{Urs Oberdorf}
\thanks{urs.oberdorf@autistici.org}
\and{Stephan Wendt}
\thanks{foo@mailinator.com}}
%\email{bar@mailinator.com}
%\institution{HTW des Saarlandes}
\date{01.07.2015}

\begin{document}
%\begin{titlepage}
%	\begin{center}
%		{\LARGE \textbf{Mein toller Titel}}
%	\end{center}
%\end{titlepage}

\maketitle

%\underline{text} %%Unterstrichener Text


%\tableofcontents

%\textwidth, \columnwidth, \linewidth

\section{Introduction}
\subsection{Angriffsmöglichkeiten}
\begin{itemize}
\item Tastatur-Eingaben (bzw. allgemein HID-Devices)
\item "Boot-Sektor-Virus"
\item gefälschter Netzwerkadapter mit DHCP-Server
\item FTP-Server
\item alles verfälschen, was auf dem Stick gespeichert werden soll
\end{itemize}




\subsection{Abwehrmöglichkeit}
\begin{itemize}
\item nur USB-Massenspeicher zulassen: darf dann aber nicht beim Reboot drin stecken, bzw. Booten von USB-Sticks unterbinden und BIOS-Passwort setzen, damit der USB-Stick nicht mit Tastatureingaben das BIOS umkonfigurieren kann.
Problem ist aber, irgendeine Tastatur muss zugelassen werden, zum Zeitpunkt
an dem das BIOS den Rechner an das OS übergibt, welche sollte das sein?
Vielleicht nach dem BIOS keine Tastatureingaben zulassen, bis eine Passworteingabe verlangt wird (entweder vom Bootloader oder vom Betriebssystem) und die Tastatur, die dann das richtige Passwort eingibt, ist die erlaubte Tastatur)
Wie würde man es mit der Maus machen?
\item Stick direkt beim Anstecken mit sauberer Firmware überschreiben? - Klappt
  wahrscheinlich nicht, weil das Betriebssystem das auf der USB-Stick-CPU
läuft, wahrscheinlich sich dazwischen schalten kann
\item The firmware of a USB device can typically only be read back with the help of that firmware (if at all): A malicious firmware can spoof a legitimate one. D.h. es würde nichts bringen, die Firmware auszulesen, um sie zu überprüfen.
Ist es irgendwie feststellbar, dass es doch mal ein USB-Stick war, bevor die manipulierte Firmware aktiv wird?
\item Abfragen, ob die Chip-Familie einen Hardware-Firmwareschreibschutz hat.
\item FreeBSD adds an option to switch off USB enumeration.
\item Vielleicht alle erlaubten USB-Geräte so umflashen, dass sie sich eindeutig identifizieren?
\item Offene Hardware und Firmware, und USB-Geräte, die man zuverlässig mit der eigenen signierten Firmware flashen kann
\end{itemize}


\subsection{Projektrichtung}
\begin{itemize}
\item neuartige Attacke im USB-Stick
\item Entwickeln und Testen von Schutzmaßnahmen
\item Versuchen, ein fest in den Laptop eingebautes Gerät zu flashen
\item Versuchen, mit forensischen Methoden einen Angriff per USB-Gerät
  festzustellen
\end{itemize}
\end{document}
